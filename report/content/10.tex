\section{Implementation}

\subsection{Generation}

Ein Ziel des Projektes war das schreiben eines Algorithmus, welcher in der Lage ist, lösbare Sudokus zu generieren.
Diese sollten eine eindeutige Lösung haben und abhängig von bestimmten Inputvariablen erstellt werden.
Zu diesen Inputvariablen zählen sowohl die Schwierigkeit, als auch bestimmte Lösungsstrategien, welche potentiell beim 
darauffolgenden Lösen durch den Nutzer eine Anwendung finden sollen.
\newline
\newline
Die Generation wurde durch drei Unterschritte realisiert. Zuerst wird versucht, ein leeres Sudoku vollständig zu füllen.
Dann werden wieder Felder gelöscht, wobei nach jedem Feld ein Prüfalgorithmus testet, ob das Sudoku noch eine eindeutige Lösung hat.
Wenn eine gewissen Anzahl von gelöschten Feldern, welche von der Schwierigkeit abhängt, erreicht wird, ist der Generationsvorgang abgeschlossen.

\subsubsection{Erzeugen eines gefüllten Sudokus}

\subsubsection{Leeren des Sudokus}

\subsubsection{Lösungsalgorithmus}